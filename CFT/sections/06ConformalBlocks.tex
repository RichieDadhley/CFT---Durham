\chapter{Conformal Blocks \& Bootstrap Programme}

\section{CFT Data (Summary)}

As we have tried to emphasise, any CFT data is characterised by the set local primary operators $\{\cO_{\Delta,I}\}$, where $I$ denotes the general Lorentz indices as before, and their correlation functions. We refer to the set $\{\cO_{\Delta,I}\}$ as the \textit{spectrum} of local primary operators.

Let's review what we know so far about the correlation functions.
\ben[label=(\roman*)] 
    \item Two-point functions: 
        \ben 
            \item Are fixed up to a normalisation, e.g. for scalars we had $\la \cO^{(1)}_{\Delta}(x)\cO^{(2)}_{\Delta}(y)\ra = \frac{C_{12}}{|x-y|^{2\Delta}}$. 
            \item We can normalise the operators, i.e. $\cO^{(i)}_{\Delta}(x) \to \cO^{(i)}_{\Delta}(x)/\sqrt{A}$, such that the 2-point functions are completely fixed.
            \item We can diagonalise our representations by the dilatation weight, i.e. pick a basis such that two different operators of the same weight are orthogonal. 
        \een 
    \item Three-point functions:
        \ben 
            \item Are fixed up to a constant $C_{123}$, e.g. for scalars we have 
            \bse
                \la \cO_{\Delta_1}(x_1) \cO_{\Delta_2}(x_2)\cO_{\Delta_3}(x_3)\ra = \frac{C_{123}}{|x_{12}|^{\Delta_1+\Delta_2-\Delta_3} |x_{23}|^{\Delta_2+\Delta_3-\Delta_1} |x_{13}|^{\Delta_1+\Delta_3-\Delta_1}}.
            \ese
            \item The constant $C_{123}$ is set by how we choose to normalise our states, in other words it is fixed once we renormalise our operators as (i)(b) above.
        \een 
    \item Higher point functions: 
        \ben 
            \item Can be reduced to three point functions using the OPE.
            \item The terms appearing in the OPE are completely fixed in terms of the three-point functions. 
        \een 
\een 

So we have seen that we can essentially completely categorise a CFT via the above steps. We therefore make the following definition. 

\bd[CFT Data]
    We call the set of the of the spectrum and the OPE coefficients the \textit{CFT data}. That is 
    \be 
    \label{eqn:CFTData}
        (\text{CFT Data}) := \{ \cO_{\Delta,I} , C_{\cO_1\cO_2\cO_3} \}.
    \ee 
\ed 

This name is fitting as given this data we can compute \textit{any} correlation function, and therefore we have solved the CFT. 

\section{Consistency Conditions On CFT Data}

The next question is what kind of CFT data gives a well defined CFT? That is can we just give \textit{any} random set and be OK? The answer is, of course, "no" and we have some consistency conditions.

The first thing we note is that we have completely glazed over unitarity so far. It turns out that unitarity imposes bounds on the allowed values of $\Delta$ for a given Lorentz rep, for example for a scalar field we have
\be 
\label{eqn:UnitarityBoundOnDelta}
    \Delta \geq \frac{D}{2}-1,
\ee 
apart from identity operator which has $\Delta=0$. The bounds for other operator types can be found in Section 7.3 of Simmons-Duffin.

\br 
    Note that a scalar field saturates this bound. This is easily seen from computing its mass dimension from the Lagrangian. 
\er 

Secondly (and more interestingly), let's consider the 4-point functions
\bse 
    \la \Phi_1(x_1)\Phi_2(x_2) \Phi_3(x_3)\Phi_4(x_4)\ra. 
\ese 
We can take an OPE between $\Phi_1(x_1)\Phi_2(x_2)$ and another OPE between $\Phi_3(x_3)\Phi_4(x_4)$ to obtain 
\bse 
    \begin{split}
        \la \Phi_1(x_1)\Phi_2(x_2) \Phi_3(x_3)\Phi_4(x_4)\ra & = \sum_{\cO,\cO^{\prime}} C_{12\cO}C_{34\cO^{\prime}} \big[ C_{\cO}(x_{12},\p_y)  C_{\cO^{\prime}}(x_{34},\p_z) \la \cO(y) \cO^{\prime}(z)\ra \big] \\
        & = \sum_{\cO} C_{12\cO} C_{34\cO} \big[ C_{\cO}(x_{12},\p_y)  C_{\cO}(x_{34},\p_z) \la \cO(y) \cO(z)\ra \big]
    \end{split}
\ese 
where the second line follow from our diagonalisation procedure, i.e. the two point function is only non-vanishing when $\cO=\cO^{\prime}$. Now we recall that the bit in square brackets is completely theory independent, as the $C_{\cO}(x,\p_y)$ are purely dynamical and only depend on the representations (i.e. weights) and similarly the two-point function only depends on $|y-z|$ and weights. We call this square bracketed terms a \textit{conformal partial wave} or \textit{conformal block}, and we shall denote it as 
\bse 
    G_{\cO}(x_1,x_2,x_3,x_4) :=  C_{\cO}(x_{12},\p_y)  C_{\cO}(x_{34},\p_z) \la \cO(y) \cO(z)\ra 
\ese
so that  
\bse 
    \la \Phi_1 ... \Phi_4 \ra = \sum_{\cO} C_{12\cO}C_{34\cO} G_{\cO}(x_1,..., x_4).
\ese 

This is good, however (as we will indeed shortly do) we could have done the OPE contractions as $\Phi_2\Phi_3$ and $\Phi_1\Phi_4$. This would change our definition of our conformal block to be\footnote{Bonus exercise, verify that this is true.} 
\bse 
    G_{\cO}(x_1,x_2,x_3,x_4) :=  C_{\cO}(x_{23},\p_y)  C_{\cO}(x_{14},\p_z) \la \cO(y) \cO(z)\ra. 
\ese 
We can therefore tweak the definition of our conformal blocks to be 
\mybox{
    \be 
    \label{eqn:ConformalBlock}
        G_{\cO}(x_1,x_2,x_3,x_4) := |x_{ij}|^{\Delta_i+\Delta_j} |x_{k\ell}|^{\Delta_k+\Delta_{\ell}} C_{\cO}(x_{ij},\p_y) C_{\cO}(x_{k\ell}, \p_z) \la \cO(y)\cO(z)\ra 
    \ee 
}
\noindent with our 4-point function then given by 
\be 
\label{eqn:FourPointFunctionConformalBlocks}
    \la \Phi_1... \Phi_4 \ra = P_{ijk\ell} \sum_{\cO} C_{ij\cO} C_{k\ell\cO} G_{\cO}(x_1,x_2,x_3x_4),
\ee 
with 
\be 
\label{eqn:Pijkl}
    P_{ijk\ell} := |x_{ij}|^{-(\Delta_i+\Delta_j)} |x_{k\ell}|^{-(\Delta_k+\Delta_{\ell})}
\ee 
This might seem like overkill, however this will come in handy shortly. 

We can depict the conformal block pictorially so that our 4-point function is given by
\begin{center}
    \btik 
        \node at (-3,0) {\Large{$\sum C_{12\cO}C_{34\cO} $}};
        \node at (-4.1,-0.5) {\small{$\cO$}};
        \draw[thick, rotate around={-45:(0,0)}] (0,0) -- (-1,0) node [left] {$1$};
        \draw[thick, rotate around={45:(0,0)}] (0,0) -- (-1,0) node [left] {$2$};
        \draw[thick, rotate around={45:(1,0)}] (1,0) -- (2,0) node [right] {$4$};
        \draw[thick, rotate around={-45:(1,0)}] (1,0) -- (2,0) node [right] {$3$};
        \draw[thick] (0,0) -- (1,0) node [midway, above] {$\cO$};
    \etik 
\end{center}

Now we did the above OPEs in $\Phi_1\Phi_2$ and $\Phi_3\Phi_4$, but we could just have easily done them for $\Phi_2\Phi_3$ and $\Phi_1\Phi_4$. If we do that we of course get a different conformal block corresponding to the following diagram
\begin{center}
    \btik 
        \node at (-3,0) {\Large{$\sum C_{23\cO}C_{14\cO} $}};
        \node at (-4.1,-0.5) {\small{$\cO$}};
        \draw[thick, rotate around={-30:(0,0.5)}] (0,0.5) -- (-1,0.5) node [left] {$1$};
        \draw[thick,rotate around={30:(0,-0.5)}] (0,-0.5) -- (-1,-0.5) node[left] {$2$};
        \draw[thick, rotate around={-30:(0,-0.5)}] (0,-0.5) -- (1,-0.5) node [right] {$3$};
        \draw[thick, rotate around={30:(0,0.5)}] (0,0.5) -- (1,0.5) node [right] {$4$};
        \draw[thick] (0,0.5) -- (0,-0.5) node [midway, right] {$\cO$};
    \etik 
\end{center}

Of course these two things have to agree (as they are both the 4-point function) and so we get a constraint between the allowed OPE coefficients $\{C_{ij\cO}\}$ where $i,j=1,2,3,4$. This relation is known as a \textit{crossing symmetry}, and it is a very powerful constraint on the allowed data by placing a constraint on the OPE associativity. 

\section{Conformal Bootstrap}

\bcl 
    Once we impose the $4$-point crossing symmetry, there are \textit{no new constraints} at higher point functions.
\ecl 

This is a non-trivial result and \textit{massively} simplifies the study of CFTs. Let's prove that it works for the case of a 5-point function. In order to lighten notation, we shall absorb the OPE coefficients into the diagrams.\footnote{They can be read off by looking at the vertices.} We shall use a $\bullet$ to indicate a point in the diagram that still needs to OPEd.

\subsubsection{Step 1: Do $\Phi_1\Phi_2$ OPE}
\begin{center}
    \btik 
        \begin{scope}
            \draw[thick, rotate around={-45:(0,0)}] (0,0) -- (-1,0) node [left] {$1$};
            \draw[thick, rotate around={45:(0,0)}] (0,0) -- (-1,0) node [left] {$2$};
            \draw[thick, rotate around={-60:(0,0)}] (0,0) -- (1,0) node [right] {$3$};
            \draw[thick] (0,0) -- (1,0) node [right] {$4$};
            \draw[thick, rotate around={60:(0,0)}] (0,0) -- (1,0) node [right] {$5$};
            \draw[fill=black] (0,0) circle [radius=0.1cm];
            \node[right] at (1.75,0) {\Large{$=$}};
        \end{scope}
        \begin{scope}[xshift=4.75cm]
            \node at (-1.75,0) {\Large{$\sum$}};
            \node at (-1.75,-0.5) {\small{$\cO$}};
            \draw[thick, rotate around={-45:(0,0)}] (0,0) -- (-1,0) node [left] {$1$};
            \draw[thick, rotate around={45:(0,0)}] (0,0) -- (-1,0) node [left] {$2$};
            \draw[thick] (0,0) -- (1,0) node [midway, above] {$\cO$};
            \draw[thick, rotate around={-60:(1,0)}] (1,0) -- (2,0) node [right] {$3$};
            \draw[thick] (1,0) -- (2,0) node [right] {$4$};
            \draw[thick, rotate around={60:(1,0)}] (1,0) -- (2,0) node [right] {$5$};
            \draw[fill=black] (1,0) circle [radius=0.1cm];
        \end{scope}
    \etik 
\end{center}

\subsubsection{Step 2: Do $\Phi_3\Phi_4$ OPE}

\begin{center}
    \btik 
        \begin{scope}
            \node at (-1.75,0) {\Large{$\sum$}};
            \node at (-1.75,-0.5) {\small{$\cO$}};
            \draw[thick, rotate around={-45:(0,0)}] (0,0) -- (-1,0) node [left] {$1$};
            \draw[thick, rotate around={45:(0,0)}] (0,0) -- (-1,0) node [left] {$2$};
            \draw[thick] (0,0) -- (1,0) node [midway, above] {$\cO$};
            \draw[thick, rotate around={-60:(1,0)}] (1,0) -- (2,0) node [right] {$3$};
            \draw[thick] (1,0) -- (2,0) node [right] {$4$};
            \draw[thick, rotate around={60:(1,0)}] (1,0) -- (2,0) node [right] {$5$};
            \draw[fill=black] (1,0) circle [radius=0.1cm];
            \node[right] at (2.75,0) {\Large{$=$}};
        \end{scope}
        \begin{scope}[xshift=5.75cm]
            \node at (-1.75,0) {\Large{$\sum$}};
            \node at (-1.75,-0.5) {\small{$\cO, \cO^{\prime}$}};
            \draw[thick, rotate around={-45:(0,0)}] (0,0) -- (-1,0) node [left] {$1$};
            \draw[thick, rotate around={45:(0,0)}] (0,0) -- (-1,0) node [left] {$2$};
            \draw[thick] (0,0) -- (1,0) node [midway, above] {$\cO$};
            \draw[thick, rotate around={60:(1,0)}] (1,0) -- (2,0) node [right] {$5$};
            \draw[thick] (1,0) -- (1.7,-0.7);
            \node at (1.6,-0.2) {$\cO^{\prime}$};
            \draw[thick, rotate around={25:(1.7,-0.7)}] (1.7,-0.7) -- (2.7,-0.7) node [right] {$4$};
            \draw[thick, rotate around={-65:(1.7,-0.7)}] (1.7,-0.7) -- (2.7,-0.7) node [right] {$3$};
        \end{scope}
    \etik 
\end{center}

\subsubsection{Step 3: Use $4$-point crossing symmetry on $\Phi_1\Phi_2\cO\Phi_5\cO^{\prime}$}

\begin{center}
    \btik 
        \begin{scope}
            \node at (-1.75,0) {\Large{$\sum$}};
            \node at (-1.75,-0.5) {\small{$\cO, \cO^{\prime}$}};
            \draw[thick, rotate around={-45:(0,0)}] (0,0) -- (-1,0) node [left] {$1$};
            \draw[thick, rotate around={45:(0,0)}] (0,0) -- (-1,0) node [left] {$2$};
            \draw[thick] (0,0) -- (1,0) node [midway, above] {$\cO$};
            \draw[thick, rotate around={60:(1,0)}] (1,0) -- (2,0) node [right] {$5$};
            \draw[thick] (1,0) -- (1.7,-0.7);
            \node at (1.6,-0.2) {$\cO^{\prime}$};
            \draw[thick, rotate around={25:(1.7,-0.7)}] (1.7,-0.7) -- (2.7,-0.7) node [right] {$4$};
            \draw[thick, rotate around={-65:(1.7,-0.7)}] (1.7,-0.7) -- (2.7,-0.7) node [right] {$3$};
            \node[right] at (3.5,0) {\Large{$=$}};
        \end{scope}
        \begin{scope}[xshift=6.75cm]
            \node at (-1.75,0) {\Large{$\sum$}};
            \node at (-1.75,-0.5) {\small{$\cO^{\prime}, \cO^{\prime\prime}$}};
            \draw[thick, rotate around={-30:(0,0.5)}] (0,0.5) -- (-1,0.5) node [left] {$1$};
            \draw[thick,rotate around={30:(0,-0.5)}] (0,-0.5) -- (-1,-0.5) node[left] {$2$};
            \draw[thick, rotate around={30:(0,0.5)}] (0,0.5) -- (1,0.5) node [right] {$5$};
            \draw[thick] (0,-0.5) -- (0.8,-1);
            \node at (0.6,-0.5) {$\cO^{\prime}$};
            \draw[thick, rotate around={25:(0.8,-1)}] (0.8,-1) -- (1.8,-1) node [right] {$4$};
            \draw[thick, rotate around={-65:(0.8,-1)}] (0.8,-1) -- (1.8,-1) node [right] {$3$};
            \draw[thick] (0,0.5) -- (0,-0.5) node [midway, left] {$\cO^{\prime\prime}$};
        \end{scope}
    \etik 
\end{center}

\subsubsection{Step 4: Use OPE on $\Phi_2\cO^{\prime}\Phi_3\Phi_4$ in reverse}

\begin{center}
    \btik 
        \begin{scope}
            \node at (-1.75,0) {\Large{$\sum$}};
            \node at (-1.75,-0.5) {\small{$\cO^{\prime}, \cO^{\prime\prime}$}};
            \draw[thick, rotate around={-30:(0,0.5)}] (0,0.5) -- (-1,0.5) node [left] {$1$};
            \draw[thick,rotate around={30:(0,-0.5)}] (0,-0.5) -- (-1,-0.5) node[left] {$2$};
            \draw[thick, rotate around={30:(0,0.5)}] (0,0.5) -- (1,0.5) node [right] {$5$};
            \draw[thick] (0,-0.5) -- (0.8,-1);
            \node at (0.6,-0.5) {$\cO^{\prime}$};
            \draw[thick, rotate around={25:(0.8,-1)}] (0.8,-1) -- (1.8,-1) node [right] {$4$};
            \draw[thick, rotate around={-65:(0.8,-1)}] (0.8,-1) -- (1.8,-1) node [right] {$3$};
            \draw[thick] (0,0.5) -- (0,-0.5) node [midway, left] {$\cO^{\prime\prime}$};
            \node[right] at (2.5,0) {\Large{$=$}};
        \end{scope}
        \begin{scope}[xshift=5.75cm]
            \node at (-1.75,0) {\Large{$\sum$}};
            \node at (-1.75,-0.5) {\small{$\cO^{\prime\prime}$}};
            \draw[thick, rotate around={-30:(0,0.5)}] (0,0.5) -- (-1,0.5) node [left] {$1$};
            \draw[thick, rotate around={30:(0,0.5)}] (0,0.5) -- (1,0.5) node [right] {$5$};
            \draw[thick,rotate around={30:(0,-0.5)}] (0,-0.5) -- (-1,-0.5) node[left] {$2$};
            \draw[thick, rotate around={10:(0,-0.5)}] (0,-0.5) -- (1,-0.5) node [right] {$4$};
            \draw[thick, rotate around={-60:(0,-0.5)}] (0,-0.5) -- (1,-0.5) node [right] {$3$};
            \draw[fill=black] (0,-0.5) circle [radius=0.1cm];
            \draw[thick] (0,0.5) -- (0,-0.5) node [midway, left] {$\cO^{\prime\prime}$};
        \end{scope}
    \etik 
\end{center}

If we then relabel $\cO^{\prime\prime}\to \cO$ and compare the right-hand side of step 1, we see that crossing at 5-points follows automatically from crossing at 4-points. 

\br 
    We did this whole thing under our assumptions that everything was a scalar. Of course it is possible to extend the idea to operators with spin. We focused on the scalar case because it is easier in practice.
\er 

This above calculation allows us to introduce the following definition for a CFT.\footnote{Multiple people have come to this conclusion, including: Ferrara, Grillo, Gatto, '73; Polyakov '74; Mack '77.}

\bd[CFT] 
   We \textit{define} a CFT as the set of data satisfying this OPE associativity. 
\ed 

This leads to what is known as the conformal \textit{bootstrap programme}, which was used to completely solve some 2d theories analytically in the 1980s (e.g. so-called \textit{minimum models}, i.e. CFTs with a finite number of primaries). In 2008 there was a programme started by Rychkov known as \textit{numerical bootstrap programme} in higher dimensions. Here the idea is to put numerical constraints on the space of theories by examining this crossing equation. 

Let's sketch the idea now.\footnote{For more details see Section 10 (in particular Section 10.4) of Simmons-Duffin.} First consider the crossing equation for 4 identical scalars of weight $\Delta$. 

\begin{center}
    \btik 
        \begin{scope}[xshift=-4cm]
            \node at (-3,0) {\Large{$\sum C_{\Phi\Phi\cO}^2 $}};
            \node at (-3.65,-0.5) {\small{$\cO$}};
            \draw[thick, rotate around={-45:(0,0)}] (0,0) -- (-1,0) node [left] {$\Phi$};
            \draw[thick, rotate around={45:(0,0)}] (0,0) -- (-1,0) node [left] {$\Phi$};
            \draw[thick, rotate around={45:(1,0)}] (1,0) -- (2,0) node [right] {$\Phi$};
            \draw[thick, rotate around={-45:(1,0)}] (1,0) -- (2,0) node [right] {$\Phi$};
            \draw[thick] (0,0) -- (1,0) node [midway, above] {$\cO$};
            \node[right] at (2.75,0) {\Large{$=$}};
        \end{scope}
        \begin{scope}[xshift=4cm]
            \node at (-3,0) {\Large{$\sum C_{\Phi\Phi\cO}^2$}};
            \node at (-3.65,-0.5) {\small{$\cO$}};
            \draw[thick, rotate around={-30:(0,0.5)}] (0,0.5) -- (-1,0.5) node [left] {$\Phi$};
            \draw[thick,rotate around={30:(0,-0.5)}] (0,-0.5) -- (-1,-0.5) node[left] {$\Phi$};
            \draw[thick, rotate around={-30:(0,-0.5)}] (0,-0.5) -- (1,-0.5) node [right] {$\Phi$};
            \draw[thick, rotate around={30:(0,0.5)}] (0,0.5) -- (1,0.5) node [right] {$\Phi$};
            \draw[thick] (0,0.5) -- (0,-0.5) node [midway, right] {$\cO$};
        \end{scope}
    \etik 
\end{center}

It is now that our redefinition \Crefrange{eqn:ConformalBlock}{eqn:Pijkl} comes in handy. As we are considering 4 identical fields our $P_{ijk\ell}$ factor simply becomes 
\bse
    P_{ijk\ell} = |x_{ij}|^{\Delta} |x_{k\ell}|^{\Delta}.
\ese
The reason this is useful is because we note that the two diagrams above correspond to 
\bse 
    P_{1234} = |x_{12}|^{\Delta}|x_{34}|^{\Delta} \qand P_{1423} = |x_{14}|^{\Delta}|x_{23}|^{\Delta},
\ese 
which we can relate to the 4-point cross-ratios, \Cref{eqn:CrossRatios}, as
\bse 
    P_{1234} = |x_{13}|^{\Delta} |x_{24}|^{\Delta} u^{\Delta} \qand P_{1423} = |x_{13}|^{\Delta} |x_{24}|^{\Delta} v^{\Delta}.
\ese 

Next note that crossing symmetry is basically the statement that the result is invariant under $1\leftrightarrow 3$ (or $2\leftrightarrow 4$), which corresponds exactly to $u\leftrightarrow v$, which is how our two $P$s are related. If we then plug all this into \Cref{eqn:FourPointFunctionConformalBlocks}, we see the above equality of diagrams is equivalent to saying 
\bse 
    \sum_{\cO} C^2_{\Phi\Phi\cO} \big[ u^{2\Delta} G_{\cO}(x_1,x_2,x_3,x_4) - v^{2\Delta} G_{\cO}(x_3,x_2,x_1,x_4)\big] = 0
\ese 
for any $u,v$. 

This is the sort of equation people play around with numerically in order to solve $D\geq 3$ CFTs. The obvious question is how does this help in that goal? Well first we notice that $C^2_{\Phi\Phi\cO} >0$. This obviously places a constraint on the allowed set of primaries, i.e. the allowed values of $C_{\Phi\Phi\cO}$. The idea is to then consider the function in square brackets as a vector, so we're asking for a sum of vectors with non-negative coefficients to vanish:
\bse 
    \sum c \underline{v} = 0
\ese 
with $c\geq 0$. The case $c=0$ just means every correlator vanishes and so is boring, so we focus on $c > 0$. Clearly this has no solutions if all the vectors live on one side of a plane as we need to cancel stuff. That is, if we want to cancel a sum of the blue and green arrows in the diagram below we need something like the red arrow. 

\begin{center}
    \btik[scale=0.8]
        \draw[ultra thick, ->, red] (4,0.5) -- (4,-1.5);
        \draw[thick, rotate around={-25:(0,0)}, xscale=1.5, fill = gray!40, opacity = 0.8] (0,0) .. controls (0.8,0.5) and (1.2,0.5) .. (3.5,1) .. controls (4,1.5) and (4,3) .. (4.5,4.5) .. controls (3.2,4) and (3.7,4) .. (1,3.5) .. controls (0.5,3) and (0.5,1.5) .. (0,0);
        \draw[thick, rotate around={-25:(0,0)}, xscale=1.5] (0,0) .. controls (0.8,0.5) and (1.2,0.5) .. (3.5,1) .. controls (4,1.5) and (4,3) .. (4.5,4.5) .. controls (3.2,4) and (3.7,4) .. (1,3.5) .. controls (0.5,3) and (0.5,1.5) .. (0,0);
        \draw[ultra thick, ->, blue, rotate around={10:(4,0.5)}] (4,0.5) -- (4,2.5);
        \draw[ultra thick, ->, green, rotate around={-40:(4,0.5)}] (4,0.5) -- (4,2.5);
        \draw[fill=black] (4,0.5) circle [radius=0.05cm];
    \etik
\end{center}

\noindent These vectors themselves are the data you are given in order to try solve the CFT. 

% For Paul: you made some comment about AdS/CFT here about how bootstrap is used to do strong coupled CFT calculations. I didn't catch everything you said in when I typed it in the lecture and I'm not confident enough with AdS/CFT to expand on this remark myself. 